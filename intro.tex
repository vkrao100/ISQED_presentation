%!TEX root = UG_PhD_Defense.tex
\begin{frame}{\large Outline}
% Overview of the problems addressed and their solutions
\bi
	\item Problem Description
	\item Motivation and Application
	\item Preliminaries
	\item Multi-Fix setup
	\bi
		\item Mathematical challenges
	\ei
	\item Rectifiability check
	\item Experimental results
	\item Conclusion and Future work
	% \bi
	% 	\item Multi-error logic rectification
	% 	\item Approximate Computing
	% \ei 
	% \item Progress \& Objectives
	% \item Rectification of finite field arithmetic circuits
	% \bi
	% 	\item Problem Modeling over Finite Fields $\Fkk$
	% 	\item Single-fix rectification of Finite field arithmetic circuits
	% 	% \bi
	% 	% 	\item Application to circuit
	% 	% \ei
	% 	\item Multi-fix rectifiability of Finite field arithmetic circuits
	% 	\bi
	% 		\item Theory and Procedures
	% 		\item Application to circuit
	% 	\ei
	% \ei
\ei
\end{frame}

\begin{frame}{\large Problem Description: Multi-error logic rectification}

\begin{figure}[hbt]
\centering
\includegraphics[scale=0.26]{mas_3_ddc_mfr_a.pdf}
\caption*{A faulty implementation of a 3-bit modulo multiplier
% ($n$=3) with gate replacement bugs introduced at nets $d_5$ (AND replaced with an OR) and $d_2$ (AND replaced with an XOR), and a wire replacement bug at net $e_0$ (input shorted to $d_0$ instead of $d_1$).
}\label{fig:mas_bug_Wa}
\end{figure}
\end{frame}


\begin{frame}{\large Problem Description}
\bi 
	% \item Identify rectification target nets
	% \bi
	% 	\item Post-verification failure analysis
	% 	\item Primary output always a candidate for a correction
	% \ei
	\item Agnostic to the fault model, check for rectification at particular targets
	\bi
		\item Single-fix Rectification (SFR)
		\bi
			\item Correct circuit by changing function at a single net
		\ei
	\ei
	\vspace{0.1in}
	\vspace{0.1in}
	\vspace{0.1in}
	\item In a general setting, SFR might not be desired or may not exist
	\bi
	\item Multi-fix Rectification (MFR)
	\bi
		\item Correct circuit by changing functions at multiple nets
		\item Contribution: Multi-fix rectifiability setup and check
	\ei
	\ei
\ei
\end{frame}

%https://ieeexplore.ieee.org/document/8405614 - cryptography

% \begin{frame}{\large Motivation: Approximate Computing}

% \bi
% 	\item Trade off accuracy for cost savings [{\it Kulkarni. et al}, VLSI'11]
% 	\item Synthesize approximate circuit by performing ECO style patch on original exact circuit
% \ei
% %\item Application to approximate circuits
% % \bi
% % 	\item Exact efficient implementation given
% % 	\item Rectify it minimally to match a new approximate $spec$
% % \ei
% \begin{figure}[hbt]
% \centering
% \includegraphics[scale = 0.25]{minterm.png}
%     \caption{Single minterm functional change. EOR = Ex-OR gate}
%     \label{fig:mint}
% \end{figure}
% \end{frame}

% \begin{frame}{\large Focus: Integer Arithmetic Circuits}
% \bi
% 	\item Approximate integer primitives
% 	\bi
% 		\item Exploits error resilience of real-world applications
% 		\item Neuromorphic Architectures, Data mining, Machine Learning, etc.
% 		\item Quantification:
% 		\bi
% 			\item Magnitude: Maximum deviation of the approximate output as
% compared to the defined constraint
% 			\item Frequency: Ratio of the number of magnitude deviations from the correct
% value to the total number of outputs
% 		\ei
% 	\ei
% 	\vspace{0.1in}
% 	\item {\it Lamb} investigated SFR of approximate integer multipliers
% 	\bi
% 		\item Single cube error (frequency) introduced across two output bits (magnitude)
% 		% \bi
% 		% 	\item Translates to corruption of a single minterm entry in the original truth table for respective outputs
% 		% \ei
% 		% \item Observations from experiments on 2/3/4-bit multiplier circuits
% 		\bi
% 			\item {\red 4\%} of 2-bit, {\red 1.6\%} of 3-bit approximate specs admits SFR
% 			\item 4-bit circuit has {\red zero} SFR points
% 		\ei
% 	\ei
% \ei
% \end{frame}



% \begin{frame}{\large Previous Approaches: Verification}
% \bi
% 	\item Canonical Decision Diagrams
% 	\bi
% 		\item BDDs, BMDs, and their variants
% 		\item Infeasible for arithmetic circuits
% 	\ei
% 	\vspace{0.1in}
% 	\item Combinational Equivalence Checking using SAT
% 	\bi
% 		\item AIG + SAT: Work for structurally similar circuits
% 	\ei
% 	\vspace{0.1in}
% 	\item Verification using computer algebra
% 	\bi
% 		\item Formulations in $\Zkk$: [Namrata et al, TCAD'07]
% 		\item Formulations in $\Fkk$: [Lv et al, TCAD’13] [Pruss et al, TCAD'16]
% 		\item Formulations in $\F_2$: [Cunxi et al, ASP-DAC'17]
% 		\item Integer arithmetic circuits: [Ciesielski et al, DAC'15] [A.Sayed et al, DATE’16] [D. Ritirc et al, FMCAD’17]
% 		[A. Sayed et al, FMCAD'16]
% 	\ei
% \ei
% \end{frame}

% \begin{frame}{\large Previous Approaches: Rectification}
% % Ideas came from ATPG, Boolean Differences, 
% % Based on Boolean reasoning engines
% % Then SAT solvers became efficient and replaced those
% % SAT based decision procedure for rectification test 
% % if it passes, the way the problem is set-up Craig interpolants
% % can be used to compute the rf
% % 12 bit: then we started looking into this problem. Our group has
% % already done on verification using computer algebra which is
% % scalable on FF ckts, from there we started looking into rectification using this formulation