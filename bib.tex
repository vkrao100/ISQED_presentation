%!TEX root = UG_PhD_Defense.tex

\begin{frame}[allowframebreaks]{\large Publications}

\begin{thebibliography}{1}
{\small

\bibitem{vikas:fmcad18}
V.~{Rao}, U.~{Gupta}, I.~{Ilioaea}, A.~{Srinath}, P.~{Kalla}, and F.~{Enescu},
  ``{Post-Verification Debugging and Rectification of Finite Field Arithmetic
  Circuits using Computer Algebra Techniques},'' in \emph{Formal Methods in
  Computer Aided Design (FMCAD)}, Oct 2018, pp. 1--9.

\bibitem{Vkrao:IWLS18}
V.~Rao, U.~Gupta, I.~Ilioaea, P.~Kalla, and F.~Enescu, ``{R}esolving {U}nknown
  {C}omponents in {A}rithmetic {C}ircuits using {C}omputer {A}lgebra {M}ethods
  - poster presentation,'' in \emph{{I}nternational {W}orkshop on {L}ogic and
  {S}ynthesis(IWLS)}, 2018.

\bibitem{utkarsh:book-chapter}
U.~Gupta, I.~Ilioaea, V.~Rao, A.~Srinath, P.~Kalla, and F.~Enescu,
  ``{Rectification of Arithmetic Circuits with Craig Interpolants in Finite
  Fields},'' in \emph{{VLSI-SoC: Design and Engineering of Electronics Systems
  Based on New Computing Paradigms}}.\hskip 1em plus 0.5em minus 0.4em\relax
  Springer International Publishing, June 2019, vol. 561, ch.~5, pp. 79--106.

\bibitem{Utkarsh:VLSI18}
------, ``{On the Rectifiability of Arithmetic Circuits using Craig
  Interpolants in Finite Fields},'' in \emph{IFIP/IEEE Intl. Conf. on VLSI
  (VLSI-SoC)}, Oct 2018, pp. 49--54.

\bibitem{utkarsh:tcad17}
U.~Gupta, P.~Kalla, and V.~Rao, ``{Boolean Gr\"obner Basis Reductions on Finite
  Field Datapath Circuits Using the Unate Cube Set Algebra},'' \emph{IEEE
  Trans. on CAD of Integrated Circuits and Systems}, vol.~38, no.~3, pp.
  576--588, Mar 2019.

\bibitem{Utkarsh:IWLS18}
U.~Gupta, I.~Ilioaea, P.~Kalla, F.~Enescu, V.~Rao, and A.~Srinath, ``{C}raig
  {I}nterpolants in {F}inite {F}ields using {A}lgebraic {G}eometry: {T}heory
  and {A}pplications,'' in \emph{Intl. Workshop on Logic and Synthesis (IWLS)},
  June 2018, pp. 70--77.

\bibitem{utkarsh:iwls17}
U.~Gupta, P.~Kalla, and V.~Rao, ``{Boolean Gr\"obner Basis Reductions on
  Datapath Circuits Using the Unate Cube Set Algebra},'' in 
  \emph{International Workshop on Logic \& Synthesis}, pp. 124--131.

}
\end{thebibliography}

% \nocite{Utkarsh:ETS19,utkarsh:book-chapter,Utkarsh:VLSI18,vikas:fmcad18,utkarsh:tcad17,Utkarsh:IWLS18,utkarsh:iwls17}

\end{frame}
