%!TEX root = UG_PhD_Defense.tex
\begin{frame}{\large Preliminaries: Finite field basics}
\bi
	\item Fields - set of elements over which operations $(+,\cdot,/)$ can be performed 
	\bi
		\item Ex. $\R,\Q,\C$
	\ei
	\vspace{0.1in}
	\item Finite fields (Galois fields) - Finite set of elements
	\bi
		\item Ex. $\F_q$, where $q=p^n,~p~=~prime,~n \in \Z_{\geq 1}$ 
		\bi
		\item When $n=1$, $\F_p = \Z_p \pmod{p}$
		\item With $p=2$, $\F_2 = \B = \{0,1\}$
		\ei
		\item On circuits, $p=2$, $n=$ data-operand width
	\ei
	\vspace{0.1in}
	\item Hardware cryptography extensively based on $\Fkn$ (we use $\Fkn$)
    % \bi
		% \item $\F_2 \subset \Fkn$, $n>1$
	% \item We are interested in fields of type $\Fkn$
	% \bi
	% 	\item Binary Galois extension fields with characteristic $2$ ($p$)
	% 	\bi
	% 		\item 
	% 		\item Bit-vector of size $n$ represents $2^n$ distinct elements
	% 	\ei
	% 	\item Finite field properties allow elegant application of algebraic geometry techniques
		% \bi
		% 	\item Practical applications to hardware design, debug, verification, and rectification
		% \ei
	% \ei
\ei
\end{frame}

\begin{frame}{\large Preliminaries: Modeling circuit polynomials over $\Fkn$}
\bi
	\item Boolean logic gates in $\F_2 ~~(\F_2 \subset \Fkn)$. Over $\F_2$, $-1=+1\pmod{2}$
\ei
	\begin{align*}
		 z &=~ \sim a &\implies &z+a+1 &\pmod{2}\\
		 z &= a \wedge b &\implies &z+a\cdot b &\pmod{2}\\
		 z &= a \vee b &\implies &z+a\cdot b + a + b &\pmod{2}\\
		 z &= a \oplus b &\implies &z+a+b &\pmod{2}
	\end{align*}

\bi
	\item Word-level polynomials [$\ga$ = primitive element of $\Fkn$]
\ei
\begin{equation*}
\begin{split}
 Output:  Z + z_0 +\gamma \cdot  z_1 + \dots +\gamma^{n-1} \cdot z_{n-1}\\
 Input:  A + a_0 +\gamma \cdot a_1 + \dots +\gamma^{n-1} \cdot a_{n-1}
\end{split}
\end{equation*}
\end{frame}

\begin{frame}{\large Problem Statement and Objective}
\bi
	\item A multivariate specification polynomial $f \in \Fkn$
	\bi
		\item $n$ is the operand width
		\item Ex. $Z = A \cdot B \pmod{P_n(x)}$ over $\Fkn$
	\ei 
	\vspace{0.1in}
	\item A faulty circuit implementation $C$ for specification $f$ 
	\bi
		\item Model gates as polynomials over $\Fkn$
	\ei
	\vspace{0.1in}
	\item A primitive polynomial $P_n(x)$ used to construct $\Fkn$
	\bi
		\item $\Fkn$ constructed as $\Fkn = \F_2[x] \pmod{ P_n(x)}$
		\item Let $\ga$ be one of the roots of $P_n(x)$, i.e. $P_n(\ga) = 0$
	\ei
	 % , s.t. $P_n(\ga) = 0$
	\vspace{0.1in}
	\item A set of $m$ targets from $C$ (modeled over $\Fkm$)
	\vspace{0.1in}
	\vspace{0.1in}
	\item Check if $C$ is rectifiable at these $m$ targets 


\ei
\end{frame}



\begin{frame}{\large Algebraic Geometry: Ideals}
\bi
	\item Let $R=\ftnwring$
	\bi
		\item $\{f_1, \dots, f_s\} \in R$
	\ei
	\vspace{0.1in}
	\item In our context
	\bi
		\item $x_1,\dots,x_d$: Variables (nets of the circuit)
		\item $Z$: bit-vector representation for variables
		\item $f_1, \dots, f_s$: Polynomials from the circuit (logic gate relations)
	\ei
	\vspace{0.1in}
	\item $J = \langle F \rangle = \langle f_1, \dots, f_s \rangle \subseteq R$
	\bi
		\item $\{h_1f_1 + \cdots + h_sf_s:~h_i \in R\}$
		\item Polynomials $f_1, \dots, f_s$: {\it basis} or {\it generators} of $J$
	\ei
	\vspace{0.1in}
	\item Vanishing Ideal: $J_0 = \langle F_0 \rangle =  \langle x_1^2+x_1,\dots,x_d^2+x_d, Z^{2^n}+Z\rangle$
	\bi
		\item Restrict solutions to $x_i$ in $\F_2$, and solutions to $Z$ in $\Fkn$
	\ei
\ei
\end{frame}

\begin{frame}{\large Algebraic Geometry: Varieties}
\bi
	% \item Let $J = \langle F \rangle = \langle f_1, \dots, f_s \rangle \subseteq R=\ftnwring$
	% \vspace{0.1in}
	% \item Vanishing Ideal: $J_0 = \langle F_0 \rangle =  \langle x_1^2+x_1,\dots,x_d^2+x_d, Z^{2^n}+Z\rangle$
	\item $J = \langle F \rangle = \langle f_1, \dots, f_s \rangle \subseteq \ftnwring $
	\vspace{0.1in}
	\item Let $\bm{a} = (a_1,\dots,a_d) \in \Fkn^d$ $s.t.$ $f_1(\bm{a}) = \cdots = f_s(\bm{a})=0$
\ei
\begin{align*}
V(J) = \text{Set of all }\{\bm{a}\} \text{ s.t. }\begin{cases}
f_1(\bm{a}) = 0, \\
f_2(\bm{a}) = 0, \\
\vdots \\
f_s(\bm{a}) = 0
\end{cases}
\end{align*}
\bi
\item $V(J)$ correspond to function mappings (Truth tables)
\ei
\end{frame}

\begin{frame}{\large \Grobner Basis and Ideal membership}
\bi
	\item An ideal $J = \langle f_1,\dots,f_s\rangle \subseteq R$ can have many generators. 
	\bi
		\item $J = \langle p_1,\dots,p_m\rangle = \dots = \langle g_1,\dots,g_t\rangle$
		\item \Grobner Basis (GB) is one such set with special properties
		% \bi
		% 	\item Presence or absence of solutions (varieties)
		% 	% \item Determine dimension of varieties.
		% 	\item Ideal membership of a polynomial
		% \ei
	\ei
	% \bi
	% 	\item Same ideal different sets of generators
	% \ei
	% \vspace{0.1in}
	
	\vspace{0.1in}
	\item Let $J =  \langle f_1,\dots,f_s\rangle = \langle g_1,\dots,g_t\rangle$
	and $G = GB(J) = \{g_1,\dots,g_t \}$.
	\bi
	\item $G$ is a \Grobner basis of $J \iff \forall f \in J, f \xrightarrow[]{g_1,\dots,g_t}_+ 0$ 
	% \ei
	% \vspace{0.1in}
	\item Ideal membership: Let $f$ be a polynomial in $R$:
	\bi
		\item if $f \xrightarrow[]{g_1,\dots,g_t}_+ 0$, then $f$ is a member of $J$. 
	\ei
	\ei
\ei
\begin{figure}
\centering
\includegraphics[scale=0.39]{GB.png}
\end{figure}
\end{frame}

% \begin{frame}{\large Extended \Grobner Basis}

% \bi
% \item Helps relate an ideal with any generating set contained in the ideal

% %\begin{center}
% \begin{equation*}\label{eqn:matrix}
% \begin{bmatrix} g_1 \\ g_2 \\ \vdots \\ g_t \end{bmatrix}  =  M \cdot
% \begin{bmatrix} f_1 \\ f_2 \\ \vdots \\ f_s \end{bmatrix}
% \end{equation*}
% \item M is a $t\times s$ matrix of polynomials

% \item if $f \in J$, then
% \bi
% \item $f = u_1g_1 + u_2g_2+ \dots+ u_tg_t$
% \item $f = v_1f_1 +\dots+v_sf_s$
% \ei
% \item Used in computation of rectification function
% \ei
% \end{frame}


%How are two ideals and their varieties related? 
%Analyzing Ideal 𝐽 is not enough Need to analyze ideal 𝐼(𝑉( 𝐽))
%SNS reasons about the ideal of polynomials that vanish on V(J).


% \begin{frame}{\large Strong Nullstellensatz}
% \bi
% \item Given an ideal $J\subset R$ and $V(J) \subseteq \Fq^d$ 
% % the {\it ideal of polynomials that vanish on} $V(J)$ is 
% \bi
% 	\item $I(V(J)) = \{ f \in R : \forall \bm{a} \in V(J), f(\bm{a}) = 0\}$.
% \ei

% \item If $f$ vanishes on $V(J)$, then $f \in I(V(J))$. 

% \item The Strong Nullstellensatz characterizes $I(V(J))$ over $\Fq$
% \bi
% 	\item Over finite fields $\Fq$, $I(V_{\Fq}(J)) = J + J_0$.
% 	\item We will use this to formulate rectification check
% \ei
% \ei
% \end{frame}